\documentclass[11pt,preprint, authoryear]{elsarticle}

\usepackage{lmodern}
%%%% My spacing
\usepackage{setspace}
\setstretch{1.2}
\DeclareMathSizes{12}{14}{10}{10}

% Wrap around which gives all figures included the [H] command, or places it "here". This can be tedious to code in Rmarkdown.
\usepackage{float}
\let\origfigure\figure
\let\endorigfigure\endfigure
\renewenvironment{figure}[1][2] {
    \expandafter\origfigure\expandafter[H]
} {
    \endorigfigure
}

\let\origtable\table
\let\endorigtable\endtable
\renewenvironment{table}[1][2] {
    \expandafter\origtable\expandafter[H]
} {
    \endorigtable
}


\usepackage{ifxetex,ifluatex}
\usepackage{fixltx2e} % provides \textsubscript
\ifnum 0\ifxetex 1\fi\ifluatex 1\fi=0 % if pdftex
  \usepackage[T1]{fontenc}
  \usepackage[utf8]{inputenc}
\else % if luatex or xelatex
  \ifxetex
    \usepackage{mathspec}
    \usepackage{xltxtra,xunicode}
  \else
    \usepackage{fontspec}
  \fi
  \defaultfontfeatures{Mapping=tex-text,Scale=MatchLowercase}
  \newcommand{\euro}{€}
\fi

\usepackage{amssymb, amsmath, amsthm, amsfonts}

\usepackage[round]{natbib}
\bibliographystyle{natbib}
\def\bibsection{\section*{References}} %%% Make "References" appear before bibliography
\usepackage{longtable}
\usepackage[margin=2.3cm,bottom=2cm,top=2.5cm, includefoot]{geometry}
\usepackage{fancyhdr}
\usepackage[bottom, hang, flushmargin]{footmisc}
\usepackage{graphicx}
\numberwithin{equation}{section}
\numberwithin{figure}{section}
\numberwithin{table}{section}
\setlength{\parindent}{0cm}
\setlength{\parskip}{1.3ex plus 0.5ex minus 0.3ex}
\usepackage{textcomp}
\renewcommand{\headrulewidth}{0.2pt}
\renewcommand{\footrulewidth}{0.3pt}

\usepackage{array}
\newcolumntype{x}[1]{>{\centering\arraybackslash\hspace{0pt}}p{#1}}

%%%%  Remove the "preprint submitted to" part. Don't worry about this either, it just looks better without it:
\journal{NOT FOR RELEASE}

 \def\tightlist{} % This allows for subbullets!

\usepackage{hyperref}
\hypersetup{breaklinks=true,
            bookmarks=true,
            colorlinks=true,
            citecolor=blue,
            urlcolor=blue,
            linkcolor=blue,
            pdfborder={0 0 0}}


% The following packages allow huxtable to work:
\usepackage{siunitx}
\usepackage{multirow}
\usepackage{hhline}
\usepackage{calc}
\usepackage{tabularx}
\usepackage{booktabs}
\usepackage{caption}
\usepackage{colortbl}

\urlstyle{same}  % don't use monospace font for urls
\setlength{\parindent}{0pt}
\setlength{\parskip}{6pt plus 2pt minus 1pt}
\setlength{\emergencystretch}{3em}  % prevent overfull lines
\setcounter{secnumdepth}{5}

%%% Use protect on footnotes to avoid problems with footnotes in titles
\let\rmarkdownfootnote\footnote%
\def\footnote{\protect\rmarkdownfootnote}
\IfFileExists{upquote.sty}{\usepackage{upquote}}{}

%%% Include extra packages specified by user
% Insert custom packages here as follows
% \usepackage{tikz}

%%% Hard setting column skips for reports - this ensures greater consistency and control over the length settings in the document.
%% page layout
%% paragraphs
\setlength{\baselineskip}{12pt plus 0pt minus 0pt}
\setlength{\parskip}{12pt plus 0pt minus 0pt}
\setlength{\parindent}{0pt plus 0pt minus 0pt}
%% floats
\setlength{\floatsep}{12pt plus 0 pt minus 0pt}
\setlength{\textfloatsep}{20pt plus 0pt minus 0pt}
\setlength{\intextsep}{14pt plus 0pt minus 0pt}
\setlength{\dbltextfloatsep}{20pt plus 0pt minus 0pt}
\setlength{\dblfloatsep}{14pt plus 0pt minus 0pt}
%% maths
\setlength{\abovedisplayskip}{12pt plus 0pt minus 0pt}
\setlength{\belowdisplayskip}{12pt plus 0pt minus 0pt}
%% lists
\setlength{\topsep}{10pt plus 0pt minus 0pt}
\setlength{\partopsep}{3pt plus 0pt minus 0pt}
\setlength{\itemsep}{5pt plus 0pt minus 0pt}
\setlength{\labelsep}{8mm plus 0mm minus 0mm}
\setlength{\parsep}{\the\parskip}
\setlength{\listparindent}{\the\parindent}
%% verbatim
\setlength{\fboxsep}{5pt plus 0pt minus 0pt}



\begin{document}

\begin{frontmatter}  %

\title{A Data Scientific Approach to Equity Backtesting Research}

\author[Add1]{Riaz J Arbi}
\ead{riazarbi@gmail.com}





\address[Add1]{University of Cape Town, Cape Town, South Africa}

\cortext[cor]{Corresponding author: Riaz J Arbi}

\begin{abstract}
\small{
Contemporary research into the cross-sectional variation in stock
returns is fraught with replication challenges. This makes it difficult
to validate important results and hampers the advance of knowledge in
the field. This project addresses some of these challenges by fitting
the backtesting research process to a data scientific workflow. By
making extensive use of standard open source statistical programming
languages the authors provide an environment in which total replication
is trivial and common statistical errors are avoided by default.
}
\end{abstract}

\vspace{1cm}

\begin{keyword}
\footnotesize{
backtest \sep data workflow\sep overfitting \sep historical simulation
\sep replication \\ \vspace{0.3cm}
\textit{JEL classification} G12
}
\end{keyword}
\vspace{0.5cm}
\end{frontmatter}



%________________________
% Header and Footers
%%%%%%%%%%%%%%%%%%%%%%%%%%%%%%%%%
\pagestyle{fancy}
\chead{}
\rhead{}
\lfoot{}
\rfoot{\footnotesize Page \thepage\\}
\lhead{}
%\rfoot{\footnotesize Page \thepage\ } % "e.g. Page 2"
\cfoot{}

%\setlength\headheight{30pt}
%%%%%%%%%%%%%%%%%%%%%%%%%%%%%%%%%
%________________________

\headsep 35pt % So that header does not go over title




\textbf{This is a working paper. It is extremely messy and not intended
to be read by anybody. It has been made publically avaialable to
facilitate discussion among collaborators on the project.}

\section{Introduction}\label{introduction}

The objective of this thesis is to rebuild the Equity Asset Management
research and trading pipeline in a Data Scientific way. The final body
of work should provide a set of best practices guidelines that Asset
Managers can use in production to improve their processes. This project
has two objectives.

\begin{itemize}
\tightlist
\item
  The first is the rigorous documentation and construction of an equity
  research environment.
\item
  The second is to engage in a process of discovery in the equity
  research space with a view to uncovering interesting topics to drill
  down to for future research.
\end{itemize}

It is expected that this body of work can sustain several thesis topics.
I have structured it in a modular way so that I can start at the
beginning of the data science stack and move towards actual analysis.
But if I don't get that far then I think the building and documentation
of an equity research environment would be of considerable use to
academics and professionals in the field.

\section{Modular Approach}\label{modular-approach}

A guiding principle in the construction of my thesis topic is ``don't
reinvent the wheel.'' My topic is structured as a series of layers. Each
successive layer is enabled by the layers below it. If a solution exists
for any particular layer, the existing in-the-wild implementation can be
documented and implemented and I can move on to the next layer. If a
layer has not been implemented, or if no readily accessible open-source
solution exists, I will endeavour to document a solution and compile a
set of best practices.

My thesis will serve several purposes. It will be -

\begin{enumerate}
\def\labelenumi{\arabic{enumi}.}
\tightlist
\item
  A collection of methods and recommendations for conducting equity
  research in conjunction with common statistical software packages.
\item
  An instruction manual for building an equity research environment to
  current statistical best practices.
\item
  A working open source system which can be cloned by equity researchers
  to enable furthering of the discipline of equity research.
\item
  A demonstration set for what can currently be achieved in the space
  using existing open source tools
\end{enumerate}

The layers, ordered according to the Data Science stack, are as follows
-

\begin{enumerate}
\def\labelenumi{\arabic{enumi}.}
\tightlist
\item
  Documentation and construction of an updatable dataset that is free
  from survivorship and look ahead bias, accurately models slippage and
  transaction costs.
\item
  Implementation of an event-based backtester that can compute optimal
  trades at each rebalancing period using both market and fundamental
  data.
\item
  Evaluation of various machine learning prediction methods for their
  alpha-generation capabilities.
\end{enumerate}

\section{Proposal}\label{proposal}

Although each of the layers mentioned above appear simple, they each
contain deep, difficult to surmount challenges. I expect that each of
these layers are sufficient for a full thesis. I would like to spend the
first four months of the 2018 calendar year building a data pipeline and
research environment. The objective of this phase is to construct a
working research environment and to search for interesting areas of
further research. In the fifth month, I will review the potential topics
I have uncovered and select one for further research. If no topics are
suitable, I will refactor the code generates in the platform building
phase, thoroughly document it, and submit it as my thesis. In this
fallback scenario my thesis will be an open source software package and
manual for the construction of a full equity research environment, form
data collection through to trading. I intend to release this as open
source software to serve as a guide for future researchers and
enthusiasts. I have already identified the following topics that could
be interesting subjects for deeper study:

\subsection{Dataset Construction}\label{dataset-construction}

Clean, accurate datasets are important for both the backtesting and
management of investment portfolios. The absence of such a dataset
severely limits the management capabilities of an asset management firm
and renders most academic research into systematic investment management
useless. By building and documenting best practices for equity dataset
construction, I hope to enable quality, real-world applicable research
in my academic peers and to provide the tools that asset managers need
to rigorously apply portfolio management theory to practice.

Equity asset managers typically measure their investments against a
benchmark. However, most asset managers obtain the benchmark timeseries
from data vendors, and if they do have the ability to create it
in-house, it is a manual human-driven process.

Data from vendors is often contradictory.

Problems to be solved in this section would include -

\subsubsection{Selection and
construction}\label{selection-and-construction}

Selection and construction of an appropriate database solution weighing
speed and simplicity against consistency and atomicity. Review of
existing technologies and schema. Research into the appropriate ways of
dealing with validation time versus transactional time and define a
method for dealing with contradictory data from multiple data sources.

\subsubsection{Automate}\label{automate}

Automate the consumption of underlying constituent data and the building
of the benchmarks from first principles. Consume multiple sources and
use selection or averaging rules to reconcile conflicting datas sources.

\subsection{Building a Backtester}\label{building-a-backtester}

Backtester construction is well documented, and there exist some
excellent open source solutions. However, these solutions tend to be
focused on US markets, and they lack certain capabilities. The primary
shortcoming of these packages is that they tend to be security-centric,
not benchmark-centric. For instance, zipine, the popular python
algorithmic trading package, currently provides no functionality for
simply executing a passive market-cap-weighted investment strategy.
Baskets of stocks can only be specified by manual lists. This renders
the zipline package useless to professional asset managers, since modern
portfolio management is centered around active-weight deviations from a
benchmark. The starting point of any portfolio should be a benchmark
portfolio, and a trading algorithm should specify deviations from the
benchmark.

Problems in this section would be -

\subsubsection{Survey}\label{survey}

A survey of existing implementations, along with a summary of
capabilities and shortcomings

\subsubsection{Assessment}\label{assessment}

Assessment of these systems from a modern portfolio theory perspective

\subsubsection{Building}\label{building}

Leverage the work done in dataset construction to build an event-driven
backtester that allows an asset manager to rigorously backtest
investment ideas.

\subsubsection{Evaluation of various Machine Learning prediction
methods}\label{evaluation-of-various-machine-learning-prediction-methods}

This topic can only be tackled once the above two topics have been dealt
with. There is a fair amount of research on the implementation of
machine learning algorithms for the prediction of stock prices. Research
in this area appears to be fraught with poor data input and inaccurate
modeling of transaction and slippage costs. This section would involve
selecting several heavily-cited papers in the space and attempting to
replicate them using our dataset and trading models. I am especially
interested in whether results are replicable, and what the impact of
real-world transaction and spread costs would be.

\subsubsection{Retooling}\label{retooling}

Retoolthe portfolio implementation and research process along the lines
of the Good Judgment Project. This would be more of a theoretical topic.
There has been a lot of good work done at the Good Judgment Project on
the process of forecasting, and on the management of professional
forecasters. See
\url{https://www.edge.org/conversation/philip_tetlock-a-short-course-in-superforecasting}.

Equity analysts and portfolio managers are professional forecasters, but
there is very little quantifiable measurement of the forecasting
abilities of these experts. Typically, analysts are simply measured on
the excess return of their stock picks relative to the universe. The
term for this is attribution, and it is the answer to the question,
``what percentage of our fund's return is attributable to each
analyst?''.

The Good Judgment Project has developed a much more fine-grained
approach to measuring forecasting, and they recognize that a forecast
can be broken into separable components. In the equity price forecasting
space, these component would be, for example, timing, direction of move,
magnitude of move, underlying driver of move. A poor stock picker may be
a very good predictor of underlying drivers of stock returns, but a poor
predictor of timing responses of stocks to drivers. This topic would
constitute a literature review of the new work done at the Good Judgment
Project and translate that work into a model framework for recording the
motivations for stock picks. Since the thesis has to be completed by the
end of 2018 I do not think I could accumulate enough data for an
evaluation of the model.

\subsubsection{Extension into a multi-year research
project}\label{extension-into-a-multi-year-research-project}

I use this research as a springboard for getting actual asset managers
to pilot the frame- work and create an ongoing research project on the
impact of this paradigm for analyst accuracy.

I could integrate the retooled research process into an automated
trading process. Trades can only happen through the bet-logging tool; no
trade occurs if any fields are null; entry and exit form the stock are
determined by the bet-logging information. If an analyst wants to extend
/ reduce / cancel a bet, a new bet must be created. For instance, a long
bet in gold can only be cancelled by a short bet in gold. This basically
makes deviation from rational investing an `opt-in' option: if no action
is taken, no behavioural biases are introduced.

The historical bets can be cubed by portfolio, analyst, time period and
stock industry. and can be plotted against the benchmark. We can use
statistics to determine who the good, well calibrated analysts are. We
can mine this data for all sorts of interesting in- sights. Does a
particular analyst generally get their initial bets correct but stymie
their bets with re-bets? Then they don't have a handle on their
behavioural biases. Does an analyst generally get their pharmaceutical
bets right, but get their iron-ore mining bets wrong? Move them away
from mining.

\section{Concluding Remarks}\label{concluding-remarks}

The topics touched upon above are by no means a comprehensive account of
what I think can be done in the space. But it should serve to convince
the reader that there is massive scope for data science in the asset
management and equity research space, both in the portfolio management
workflow and in research. I would argue, in fact, that there is a
significant overlap in the work that equity analysts do and that which
data scientists do.

Too often a finance graduate glosses past the data collection, cleaning
and modeling phase of their project in order to get to their problem of
interest. This invariably results in results which are worthless in the
wild and is primary contributor to the maxim that ``every academic
backtest yields alpha, but none actually makes money''. My singular
conviction is that equity researchers need to be familiarized with data
science tools; they need data science-grade environments and data
sources. I believe that it would be a tremendous contribution to this
field if I rigorously document how a researcher can make that
transition. A secondary goal is to actually implement the environment
and then start using the tools of data science to solve problems in the
areas of asset management and equity research.

% Force include bibliography in my chosen format:
\newpage
\nocite{*}
\bibliography{}





\end{document}
