\documentclass[11pt,preprint, authoryear]{elsarticle}

\usepackage{lmodern}
%%%% My spacing
\usepackage{setspace}
\setstretch{1.2}
\DeclareMathSizes{12}{14}{10}{10}

% Wrap around which gives all figures included the [H] command, or places it "here". This can be tedious to code in Rmarkdown.
\usepackage{float}
\let\origfigure\figure
\let\endorigfigure\endfigure
\renewenvironment{figure}[1][2] {
    \expandafter\origfigure\expandafter[H]
} {
    \endorigfigure
}

\let\origtable\table
\let\endorigtable\endtable
\renewenvironment{table}[1][2] {
    \expandafter\origtable\expandafter[H]
} {
    \endorigtable
}


\usepackage{ifxetex,ifluatex}
\usepackage{fixltx2e} % provides \textsubscript
\ifnum 0\ifxetex 1\fi\ifluatex 1\fi=0 % if pdftex
  \usepackage[T1]{fontenc}
  \usepackage[utf8]{inputenc}
\else % if luatex or xelatex
  \ifxetex
    \usepackage{mathspec}
    \usepackage{xltxtra,xunicode}
  \else
    \usepackage{fontspec}
  \fi
  \defaultfontfeatures{Mapping=tex-text,Scale=MatchLowercase}
  \newcommand{\euro}{€}
\fi

\usepackage{amssymb, amsmath, amsthm, amsfonts}

\usepackage[round]{natbib}
\bibliographystyle{natbib}
\def\bibsection{\section*{References}} %%% Make "References" appear before bibliography
\usepackage{longtable}
\usepackage[margin=2.3cm,bottom=2cm,top=2.5cm, includefoot]{geometry}
\usepackage{fancyhdr}
\usepackage[bottom, hang, flushmargin]{footmisc}
\usepackage{graphicx}
\numberwithin{equation}{section}
\numberwithin{figure}{section}
\numberwithin{table}{section}
\setlength{\parindent}{0cm}
\setlength{\parskip}{1.3ex plus 0.5ex minus 0.3ex}
\usepackage{textcomp}
\renewcommand{\headrulewidth}{0.2pt}
\renewcommand{\footrulewidth}{0.3pt}

\usepackage{array}
\newcolumntype{x}[1]{>{\centering\arraybackslash\hspace{0pt}}p{#1}}

%%%%  Remove the "preprint submitted to" part. Don't worry about this either, it just looks better without it:
\makeatletter
\def\ps@pprintTitle{%
  \let\@oddhead\@empty
  \let\@evenhead\@empty
  \let\@oddfoot\@empty
  \let\@evenfoot\@oddfoot
}
\makeatother

 \def\tightlist{} % This allows for subbullets!

\usepackage{hyperref}
\hypersetup{breaklinks=true,
            bookmarks=true,
            colorlinks=true,
            citecolor=blue,
            urlcolor=blue,
            linkcolor=blue,
            pdfborder={0 0 0}}


% The following packages allow huxtable to work:
\usepackage{siunitx}
\usepackage{multirow}
\usepackage{hhline}
\usepackage{calc}
\usepackage{tabularx}
\usepackage{booktabs}
\usepackage{caption}
\usepackage{colortbl}

\urlstyle{same}  % don't use monospace font for urls
\setlength{\parindent}{0pt}
\setlength{\parskip}{6pt plus 2pt minus 1pt}
\setlength{\emergencystretch}{3em}  % prevent overfull lines
\setcounter{secnumdepth}{5}

%%% Use protect on footnotes to avoid problems with footnotes in titles
\let\rmarkdownfootnote\footnote%
\def\footnote{\protect\rmarkdownfootnote}
\IfFileExists{upquote.sty}{\usepackage{upquote}}{}

%%% Include extra packages specified by user
% Insert custom packages here as follows
% \usepackage{tikz}

%%% Hard setting column skips for reports - this ensures greater consistency and control over the length settings in the document.
%% page layout
%% paragraphs
\setlength{\baselineskip}{12pt plus 0pt minus 0pt}
\setlength{\parskip}{12pt plus 0pt minus 0pt}
\setlength{\parindent}{0pt plus 0pt minus 0pt}
%% floats
\setlength{\floatsep}{12pt plus 0 pt minus 0pt}
\setlength{\textfloatsep}{20pt plus 0pt minus 0pt}
\setlength{\intextsep}{14pt plus 0pt minus 0pt}
\setlength{\dbltextfloatsep}{20pt plus 0pt minus 0pt}
\setlength{\dblfloatsep}{14pt plus 0pt minus 0pt}
%% maths
\setlength{\abovedisplayskip}{12pt plus 0pt minus 0pt}
\setlength{\belowdisplayskip}{12pt plus 0pt minus 0pt}
%% lists
\setlength{\topsep}{10pt plus 0pt minus 0pt}
\setlength{\partopsep}{3pt plus 0pt minus 0pt}
\setlength{\itemsep}{5pt plus 0pt minus 0pt}
\setlength{\labelsep}{8mm plus 0mm minus 0mm}
\setlength{\parsep}{\the\parskip}
\setlength{\listparindent}{\the\parindent}
%% verbatim
\setlength{\fboxsep}{5pt plus 0pt minus 0pt}



\begin{document}

\begin{frontmatter}  %

\title{Helping You Write Academic Papers in R using Texevier}

\author[Add1]{Nico Katzke}
\ead{nfkatzke@gmail.com}

\author[Add1,Add2]{John Smith}
\ead{John Smith@gmail.com}

\author[Add1,Add2]{John Doe}
\ead{JohnSmith@gmail.com}



\address[Add1]{Prescient Securities, Cape Town, South Africa}
\address[Add2]{Some other Institution, Cape Town, South Africa}

\cortext[cor]{Corresponding author: Nico Katzke}

\begin{abstract}
\small{
Abstract to be written here. The abstract should not be too long and
should provide the reader with a good understanding what you are writing
about. Academic papers are not like novels where you keep the reader in
suspense. To be effective in getting others to read your paper, be as
open and concise about your findings here as possible. Ideally, upon
reading your abstract, the reader should feel he / she must read your
paper in entirety.
}
\end{abstract}

\vspace{1cm}

\begin{keyword}
\footnotesize{
Multivariate GARCH \sep Kalman Filter \sep Copula \\ \vspace{0.3cm}
\textit{JEL classification} L250 \sep L100
}
\end{keyword}
\vspace{0.5cm}
\end{frontmatter}



%________________________
% Header and Footers
%%%%%%%%%%%%%%%%%%%%%%%%%%%%%%%%%
\pagestyle{fancy}
\chead{}
\rhead{}
\lfoot{}
\rfoot{\footnotesize Page \thepage\\}
\lhead{}
%\rfoot{\footnotesize Page \thepage\ } % "e.g. Page 2"
\cfoot{}

%\setlength\headheight{30pt}
%%%%%%%%%%%%%%%%%%%%%%%%%%%%%%%%%
%________________________

\headsep 35pt % So that header does not go over title




\section{This is the Proof-of-Concept Document. It will evolve into my
Dissertation.}\label{this-is-the-proof-of-concept-document.-it-will-evolve-into-my-dissertation.}

\section{\texorpdfstring{Introduction
\label{Introduction}}{Introduction }}\label{introduction}

References are to be made as follows: Fama and French
(\protect\hyperlink{ref-fama1997}{1997}, 33) and Grinold and Kahn
(\protect\hyperlink{ref-grinold2000}{2000}) Such authors could also be
referenced in brackets (Grinold and Kahn
\protect\hyperlink{ref-grinold2000}{2000}) and together (Fama and French
\protect\hyperlink{ref-fama1997}{1997} \& Grinold and Kahn
(\protect\hyperlink{ref-grinold2000}{2000})). Source the reference code
from scholar.google.com by clicking on ``cite'' below article name. Then
select BibTeX at the bottom of the Cite window, and proceed to copy and
paste this code into your ref.bib file, located in the directory's Tex
folder. Open this file in Rstudio for ease of management, else open it
in your preferred Tex environment. Add and manage your article details
here for simplicity - once saved, it will self-adjust in your paper.

\begin{quote}
I suggest renaming the top line after @article, as done in the template
ref.bib file, to something more intuitive for you to remember. Do not
change the rest of the code. Also, be mindful of the fact that bib
references from google scholar may at times be incorrect. Reference
Latex forums for correct bibtex notation.
\end{quote}

To reference a section, you have to set a label using
``\textbackslash{}label'' in R, and then reference it in-text as e.g.:
section \ref{Data}.

Writing in Rmarkdown is surprizingly easy - see
\href{https://www.rstudio.com/wp-content/uploads/2015/03/rmarkdown-reference.pdf}{this
website} cheatsheet for a summary on writing Rmd writing tips.

\section{\texorpdfstring{Data \label{Data}}{Data }}\label{data}

Discussion of data should be thorough with a table of statistics and
ideally a figure.

In your tempalte folder, you will find a Data and a Code folder. In
order to keep your data files neat, store all of them in your Data
folder. Also, I strongly suggest keeping this Rmd file for writing and
executing commands, not writing out long pieces of data-wrangling. In
the example below, I simply create a ggplot template for scatter plot
consistency. I suggest keeping all your data in a data folder.

\begin{figure}[H]

{\centering \includegraphics{dissertation_WIP_files/figure-latex/Figure1-1} 

}

\caption{Caption Here \label{Figure1}}\label{fig:Figure1}
\end{figure}

To reference the plot above, add a ``\textbackslash{}label'' after the
caption in the chunk heading, as done above. Then reference the plot as
such: As can be seen, figure \ref{Figure1} is excellent. The nice thing
now is that it correctly numbers all your figures (and sections or
tables) and will update if it moves. The links are also dynamic.

I very strongly suggest using ggplot2 (ideally in combination with
dplyr) using the ggtheme package to change the themes of your figures.

Also note the information that I have placed above the chunks in the
code chunks for the figures. You can edit any of these easily - visit
the Rmarkdown webpage for more information.

Here follows another figure from built-in ggplot2 data:

\begin{figure}[H]

{\centering \includegraphics{dissertation_WIP_files/figure-latex/figure2-1} 

}

\caption{Diamond Cut Plot \label{lit}}\label{fig:figure2}
\end{figure}

\section{Methodology}\label{methodology}

\subsection{Subsection}\label{subsection}

Ideally do not overuse subsections. It equates to bad writing.\footnote{This
  is an example of a footnote by the way. Something that should also not
  be overused.}

\subsection{Math section}\label{math-section}

Equations should be written as such:

\begin{align} 
\beta = \sum_{i = 1}^{\infty}\frac{\alpha^2}{\sigma_{t-1}^2} \label{eq1} \\ 
\int_{x = 1}^{\infty}x_{i} = 1 \notag
\end{align}

If you would like to see the equations as you type in Rmarkdown, use \$
symbols instead (see this for yourself by adjusted the equation):

\[
\beta = \sum_{i = 1}^{\infty}\frac{\alpha^2}{\sigma_{t-1}^2} \\ 
\int_{x = 1}^{\infty}x_{i} = 1
\]

Note again the reference to equation \ref{eq1}. Writing nice math
requires practice. Note I used a forward slashes to make a space in the
equations. I can also align equations using \textbf{\&}, and set to
numbering only the first line. Now I will have to type ``begin
equation'' which is a native \LaTeX command. Here follows a more
complicated equation:

\begin{align} 
    y_t &= c + B(L) y_{t-1} + e_t   \label{eq2}    \\ \notag 
    e_t &= H_t^{1/2}  z_t ; \quad z_t \sim  N(0,I_N) \quad \& \quad H_t = D_tR_tD_t \\ \notag
        D_t^2 &= {\sigma_{1,t}, \dots, \sigma_{N,t}}   \\ \notag
        \sigma_{i,t}^2 &= \gamma_i+\kappa_{i,t}  v_{i, t-1}^2 +\eta_i  \sigma_{i, t-1}^2, \quad \forall i \\ \notag
        R_{t, i, j} &= {diag(Q_{t, i, j}}^{-1}) . Q_{t, i, j} . diag(Q_{t, i, j}^{-1})  \\ \notag
        Q_{t, i, j} &= (1-\alpha-\beta)  \bar{Q} + \alpha  z_t  z_t'  + \beta  Q_{t, i, j} \notag
\end{align}

Note that in \ref{eq2} I have aligned the equations by the equal signs.
I also want only one tag, and I create spaces using ``quads''.

See if you can figure out how to do complex math using the two examples
provided in \ref{eq1} and \ref{eq2}.

\section{Results}\label{results}

Tables can be included as follows. Use the \emph{xtable} (or kable)
package for tables. Table placement = H implies Latex tries to place the
table Here, and not on a new page (there are, however, very many ways to
skin this cat. Luckily there are many forums online!).

\begin{table}[H]
\centering
\begin{tabular}{rrrrrrrrrrrr}
  \hline
 & mpg & cyl & disp & hp & drat & wt & qsec & vs & am & gear & carb \\ 
  \hline
Mazda RX4 & 21.00 & 6.00 & 160.00 & 110.00 & 3.90 & 2.62 & 16.46 & 0.00 & 1.00 & 4.00 & 4.00 \\ 
  Mazda RX4 Wag & 21.00 & 6.00 & 160.00 & 110.00 & 3.90 & 2.88 & 17.02 & 0.00 & 1.00 & 4.00 & 4.00 \\ 
  Datsun 710 & 22.80 & 4.00 & 108.00 & 93.00 & 3.85 & 2.32 & 18.61 & 1.00 & 1.00 & 4.00 & 1.00 \\ 
  Hornet 4 Drive & 21.40 & 6.00 & 258.00 & 110.00 & 3.08 & 3.21 & 19.44 & 1.00 & 0.00 & 3.00 & 1.00 \\ 
  Hornet Sportabout & 18.70 & 8.00 & 360.00 & 175.00 & 3.15 & 3.44 & 17.02 & 0.00 & 0.00 & 3.00 & 2.00 \\ 
   \hline
\end{tabular}
\caption{Short Table Example \label{tab1}} 
\end{table}

To reference calculations \textbf{in text}, \emph{do this:} From table
\ref{tab1} we see the average value of mpg is 20.98.

Including tables that span across pages, use the following (note that I
add below the table: ``continue on the next page''). This is a neat way
of splitting your table across a page.

Use the following default settings to build your own possibly long
tables. Note that the following will fit on one page if it can, but
cleanly spreads over multiple pages:

\begingroup\fontsize{12pt}{13pt}\selectfont

\begin{longtable}{rrrrrrrrrrr}
  \toprule
mpg & cyl & disp & hp & drat & wt & qsec & vs & am & gear & carb \\ 
  \hline 
\endhead 
\hline 
{\footnotesize Continued on next page} 
\endfoot 
\endlastfoot 
 \midrule
21.00 & 6.00 & 160.00 & 110.00 & 3.90 & 2.62 & 16.46 & 0.00 & 1.00 & 4.00 & 4.00 \\ 
  21.00 & 6.00 & 160.00 & 110.00 & 3.90 & 2.88 & 17.02 & 0.00 & 1.00 & 4.00 & 4.00 \\ 
  22.80 & 4.00 & 108.00 & 93.00 & 3.85 & 2.32 & 18.61 & 1.00 & 1.00 & 4.00 & 1.00 \\ 
  21.40 & 6.00 & 258.00 & 110.00 & 3.08 & 3.21 & 19.44 & 1.00 & 0.00 & 3.00 & 1.00 \\ 
  18.70 & 8.00 & 360.00 & 175.00 & 3.15 & 3.44 & 17.02 & 0.00 & 0.00 & 3.00 & 2.00 \\ 
  18.10 & 6.00 & 225.00 & 105.00 & 2.76 & 3.46 & 20.22 & 1.00 & 0.00 & 3.00 & 1.00 \\ 
  14.30 & 8.00 & 360.00 & 245.00 & 3.21 & 3.57 & 15.84 & 0.00 & 0.00 & 3.00 & 4.00 \\ 
  24.40 & 4.00 & 146.70 & 62.00 & 3.69 & 3.19 & 20.00 & 1.00 & 0.00 & 4.00 & 2.00 \\ 
  22.80 & 4.00 & 140.80 & 95.00 & 3.92 & 3.15 & 22.90 & 1.00 & 0.00 & 4.00 & 2.00 \\ 
  19.20 & 6.00 & 167.60 & 123.00 & 3.92 & 3.44 & 18.30 & 1.00 & 0.00 & 4.00 & 4.00 \\ 
  17.80 & 6.00 & 167.60 & 123.00 & 3.92 & 3.44 & 18.90 & 1.00 & 0.00 & 4.00 & 4.00 \\ 
  16.40 & 8.00 & 275.80 & 180.00 & 3.07 & 4.07 & 17.40 & 0.00 & 0.00 & 3.00 & 3.00 \\ 
  17.30 & 8.00 & 275.80 & 180.00 & 3.07 & 3.73 & 17.60 & 0.00 & 0.00 & 3.00 & 3.00 \\ 
  15.20 & 8.00 & 275.80 & 180.00 & 3.07 & 3.78 & 18.00 & 0.00 & 0.00 & 3.00 & 3.00 \\ 
  10.40 & 8.00 & 472.00 & 205.00 & 2.93 & 5.25 & 17.98 & 0.00 & 0.00 & 3.00 & 4.00 \\ 
  10.40 & 8.00 & 460.00 & 215.00 & 3.00 & 5.42 & 17.82 & 0.00 & 0.00 & 3.00 & 4.00 \\ 
  14.70 & 8.00 & 440.00 & 230.00 & 3.23 & 5.34 & 17.42 & 0.00 & 0.00 & 3.00 & 4.00 \\ 
  32.40 & 4.00 & 78.70 & 66.00 & 4.08 & 2.20 & 19.47 & 1.00 & 1.00 & 4.00 & 1.00 \\ 
  30.40 & 4.00 & 75.70 & 52.00 & 4.93 & 1.61 & 18.52 & 1.00 & 1.00 & 4.00 & 2.00 \\ 
  33.90 & 4.00 & 71.10 & 65.00 & 4.22 & 1.83 & 19.90 & 1.00 & 1.00 & 4.00 & 1.00 \\ 
  21.50 & 4.00 & 120.10 & 97.00 & 3.70 & 2.46 & 20.01 & 1.00 & 0.00 & 3.00 & 1.00 \\ 
  15.50 & 8.00 & 318.00 & 150.00 & 2.76 & 3.52 & 16.87 & 0.00 & 0.00 & 3.00 & 2.00 \\ 
  15.20 & 8.00 & 304.00 & 150.00 & 3.15 & 3.44 & 17.30 & 0.00 & 0.00 & 3.00 & 2.00 \\ 
  13.30 & 8.00 & 350.00 & 245.00 & 3.73 & 3.84 & 15.41 & 0.00 & 0.00 & 3.00 & 4.00 \\ 
  19.20 & 8.00 & 400.00 & 175.00 & 3.08 & 3.85 & 17.05 & 0.00 & 0.00 & 3.00 & 2.00 \\ 
  27.30 & 4.00 & 79.00 & 66.00 & 4.08 & 1.94 & 18.90 & 1.00 & 1.00 & 4.00 & 1.00 \\ 
  26.00 & 4.00 & 120.30 & 91.00 & 4.43 & 2.14 & 16.70 & 0.00 & 1.00 & 5.00 & 2.00 \\ 
  30.40 & 4.00 & 95.10 & 113.00 & 3.77 & 1.51 & 16.90 & 1.00 & 1.00 & 5.00 & 2.00 \\ 
  15.80 & 8.00 & 351.00 & 264.00 & 4.22 & 3.17 & 14.50 & 0.00 & 1.00 & 5.00 & 4.00 \\ 
  19.70 & 6.00 & 145.00 & 175.00 & 3.62 & 2.77 & 15.50 & 0.00 & 1.00 & 5.00 & 6.00 \\ 
  15.00 & 8.00 & 301.00 & 335.00 & 3.54 & 3.57 & 14.60 & 0.00 & 1.00 & 5.00 & 8.00 \\ 
  21.40 & 4.00 & 121.00 & 109.00 & 4.11 & 2.78 & 18.60 & 1.00 & 1.00 & 4.00 & 2.00 \\ 
   \bottomrule
\caption{Long Table Example} 
\end{longtable}

\endgroup

\hfill

\subsection{Huxtable}\label{huxtable}

Huxtable is a very nice package for making working with tables between
Rmarkdown and Tex easier.

This cost some adjustment to the Tex templates to make it work, but it
now works nicely.

See documentation for this package
\href{https://hughjonesd.github.io/huxtable/huxtable.html}{here}. A
particularly nice addition of this package is for making the printing of
regression results a joy (see
\href{https://hughjonesd.github.io/huxtable/huxtable.html\#creating-a-regression-table}{here}).
Here follows an example:

\begin{table}[h]
\centering\captionsetup{justification=centering,singlelinecheck=off}
\caption{Regression Output}
\label{Reg01}
\begin{tabularx}{0.6\textwidth}{p{0.12\textwidth} p{0.12\textwidth} p{0.12\textwidth} p{0.12\textwidth} p{0.12\textwidth}}
\multicolumn{1}{!{\color[RGB]{0, 0, 0}\vrule width 0pt}c!{\color[RGB]{0, 0, 0}\vrule width 0pt}}{\hspace*{4pt}\rule{0pt}{\baselineskip+4pt}\centering {\fontsize{10pt}{12pt}\selectfont }\rule[-4pt]{0pt}{4pt}\hspace*{4pt}} & 
\multicolumn{1}{c!{\color[RGB]{0, 0, 0}\vrule width 0pt}}{\hspace*{4pt}\rule{0pt}{\baselineskip+4pt}\centering {\fontsize{10pt}{12pt}\selectfont Reg1}\rule[-4pt]{0pt}{4pt}\hspace*{4pt}} & 
\multicolumn{1}{c!{\color[RGB]{0, 0, 0}\vrule width 0pt}}{\hspace*{4pt}\rule{0pt}{\baselineskip+4pt}\centering {\fontsize{10pt}{12pt}\selectfont Reg2}\rule[-4pt]{0pt}{4pt}\hspace*{4pt}} & 
\multicolumn{1}{c!{\color[RGB]{0, 0, 0}\vrule width 0pt}}{\hspace*{4pt}\rule{0pt}{\baselineskip+4pt}\centering {\fontsize{10pt}{12pt}\selectfont Reg3}\rule[-4pt]{0pt}{4pt}\hspace*{4pt}} & 
\multicolumn{1}{c!{\color[RGB]{0, 0, 0}\vrule width 0pt}}{\hspace*{4pt}\rule{0pt}{\baselineskip+4pt}\centering {\fontsize{10pt}{12pt}\selectfont Reg4}\rule[-4pt]{0pt}{4pt}\hspace*{4pt}}\tabularnewline[-0.5pt]


\hhline{>{\arrayrulecolor[RGB]{0, 0, 0}\global\arrayrulewidth=0.4pt}->{\arrayrulecolor[RGB]{0, 0, 0}\global\arrayrulewidth=0.4pt}->{\arrayrulecolor[RGB]{0, 0, 0}\global\arrayrulewidth=0.4pt}->{\arrayrulecolor[RGB]{0, 0, 0}\global\arrayrulewidth=0.4pt}->{\arrayrulecolor[RGB]{0, 0, 0}\global\arrayrulewidth=0.4pt}-}
\arrayrulecolor{black}
\multicolumn{1}{!{\color[RGB]{0, 0, 0}\vrule width 0pt}l!{\color[RGB]{0, 0, 0}\vrule width 0pt}}{\hspace*{4pt}\rule{0pt}{\baselineskip+4pt}\raggedright {\fontsize{10pt}{12pt}\selectfont (Intercept)}\rule[-4pt]{0pt}{4pt}\hspace*{4pt}} & 
\multicolumn{1}{r!{\color[RGB]{0, 0, 0}\vrule width 0pt}}{\hspace*{4pt}\rule{0pt}{\baselineskip+4pt}\raggedleft {\fontsize{10pt}{12pt}\selectfont -2256.361 ***}\rule[-4pt]{0pt}{4pt}\hspace*{4pt}} & 
\multicolumn{1}{r!{\color[RGB]{0, 0, 0}\vrule width 0pt}}{\hspace*{4pt}\rule{0pt}{\baselineskip+4pt}\raggedleft {\fontsize{10pt}{12pt}\selectfont 5763.668 ***}\rule[-4pt]{0pt}{4pt}\hspace*{4pt}} & 
\multicolumn{1}{r!{\color[RGB]{0, 0, 0}\vrule width 0pt}}{\hspace*{4pt}\rule{0pt}{\baselineskip+4pt}\raggedleft {\fontsize{10pt}{12pt}\selectfont 4045.333 ***}\rule[-4pt]{0pt}{4pt}\hspace*{4pt}} & 
\multicolumn{1}{r!{\color[RGB]{0, 0, 0}\vrule width 0pt}}{\hspace*{4pt}\rule{0pt}{\baselineskip+4pt}\raggedleft {\fontsize{10pt}{12pt}\selectfont -7823.738 ***}\rule[-4pt]{0pt}{4pt}\hspace*{4pt}}\tabularnewline[-0.5pt]
\multicolumn{1}{!{\color[RGB]{0, 0, 0}\vrule width 0pt}l!{\color[RGB]{0, 0, 0}\vrule width 0pt}}{\hspace*{4pt}\rule{0pt}{\baselineskip+4pt}\raggedright {\fontsize{10pt}{12pt}\selectfont }\rule[-4pt]{0pt}{4pt}\hspace*{4pt}} & 
\multicolumn{1}{r!{\color[RGB]{0, 0, 0}\vrule width 0pt}}{\hspace*{4pt}\rule{0pt}{\baselineskip+4pt}\raggedleft {\fontsize{10pt}{12pt}\selectfont (13.055)~~~}\rule[-4pt]{0pt}{4pt}\hspace*{4pt}} & 
\multicolumn{1}{r!{\color[RGB]{0, 0, 0}\vrule width 0pt}}{\hspace*{4pt}\rule{0pt}{\baselineskip+4pt}\raggedleft {\fontsize{10pt}{12pt}\selectfont (740.556)~~~}\rule[-4pt]{0pt}{4pt}\hspace*{4pt}} & 
\multicolumn{1}{r!{\color[RGB]{0, 0, 0}\vrule width 0pt}}{\hspace*{4pt}\rule{0pt}{\baselineskip+4pt}\raggedleft {\fontsize{10pt}{12pt}\selectfont (286.205)~~~}\rule[-4pt]{0pt}{4pt}\hspace*{4pt}} & 
\multicolumn{1}{r!{\color[RGB]{0, 0, 0}\vrule width 0pt}}{\hspace*{4pt}\rule{0pt}{\baselineskip+4pt}\raggedleft {\fontsize{10pt}{12pt}\selectfont (592.049)~~~}\rule[-4pt]{0pt}{4pt}\hspace*{4pt}}\tabularnewline[-0.5pt]
\multicolumn{1}{!{\color[RGB]{0, 0, 0}\vrule width 0pt}l!{\color[RGB]{0, 0, 0}\vrule width 0pt}}{\hspace*{4pt}\rule{0pt}{\baselineskip+4pt}\raggedright {\fontsize{10pt}{12pt}\selectfont carat}\rule[-4pt]{0pt}{4pt}\hspace*{4pt}} & 
\multicolumn{1}{r!{\color[RGB]{0, 0, 0}\vrule width 0pt}}{\hspace*{4pt}\rule{0pt}{\baselineskip+4pt}\raggedleft {\fontsize{10pt}{12pt}\selectfont 7756.426 ***}\rule[-4pt]{0pt}{4pt}\hspace*{4pt}} & 
\multicolumn{1}{r!{\color[RGB]{0, 0, 0}\vrule width 0pt}}{\hspace*{4pt}\rule{0pt}{\baselineskip+4pt}\raggedleft {\fontsize{10pt}{12pt}\selectfont ~~~~~~~~}\rule[-4pt]{0pt}{4pt}\hspace*{4pt}} & 
\multicolumn{1}{r!{\color[RGB]{0, 0, 0}\vrule width 0pt}}{\hspace*{4pt}\rule{0pt}{\baselineskip+4pt}\raggedleft {\fontsize{10pt}{12pt}\selectfont 7765.141 ***}\rule[-4pt]{0pt}{4pt}\hspace*{4pt}} & 
\multicolumn{1}{r!{\color[RGB]{0, 0, 0}\vrule width 0pt}}{\hspace*{4pt}\rule{0pt}{\baselineskip+4pt}\raggedleft {\fontsize{10pt}{12pt}\selectfont 20742.600 ***}\rule[-4pt]{0pt}{4pt}\hspace*{4pt}}\tabularnewline[-0.5pt]
\multicolumn{1}{!{\color[RGB]{0, 0, 0}\vrule width 0pt}l!{\color[RGB]{0, 0, 0}\vrule width 0pt}}{\hspace*{4pt}\rule{0pt}{\baselineskip+4pt}\raggedright {\fontsize{10pt}{12pt}\selectfont }\rule[-4pt]{0pt}{4pt}\hspace*{4pt}} & 
\multicolumn{1}{r!{\color[RGB]{0, 0, 0}\vrule width 0pt}}{\hspace*{4pt}\rule{0pt}{\baselineskip+4pt}\raggedleft {\fontsize{10pt}{12pt}\selectfont (14.067)~~~}\rule[-4pt]{0pt}{4pt}\hspace*{4pt}} & 
\multicolumn{1}{r!{\color[RGB]{0, 0, 0}\vrule width 0pt}}{\hspace*{4pt}\rule{0pt}{\baselineskip+4pt}\raggedleft {\fontsize{10pt}{12pt}\selectfont ~~~~~~~~}\rule[-4pt]{0pt}{4pt}\hspace*{4pt}} & 
\multicolumn{1}{r!{\color[RGB]{0, 0, 0}\vrule width 0pt}}{\hspace*{4pt}\rule{0pt}{\baselineskip+4pt}\raggedleft {\fontsize{10pt}{12pt}\selectfont (14.009)~~~}\rule[-4pt]{0pt}{4pt}\hspace*{4pt}} & 
\multicolumn{1}{r!{\color[RGB]{0, 0, 0}\vrule width 0pt}}{\hspace*{4pt}\rule{0pt}{\baselineskip+4pt}\raggedleft {\fontsize{10pt}{12pt}\selectfont (567.672)~~~}\rule[-4pt]{0pt}{4pt}\hspace*{4pt}}\tabularnewline[-0.5pt]
\multicolumn{1}{!{\color[RGB]{0, 0, 0}\vrule width 0pt}l!{\color[RGB]{0, 0, 0}\vrule width 0pt}}{\hspace*{4pt}\rule{0pt}{\baselineskip+4pt}\raggedright {\fontsize{10pt}{12pt}\selectfont depth}\rule[-4pt]{0pt}{4pt}\hspace*{4pt}} & 
\multicolumn{1}{r!{\color[RGB]{0, 0, 0}\vrule width 0pt}}{\hspace*{4pt}\rule{0pt}{\baselineskip+4pt}\raggedleft {\fontsize{10pt}{12pt}\selectfont ~~~~~~~~}\rule[-4pt]{0pt}{4pt}\hspace*{4pt}} & 
\multicolumn{1}{r!{\color[RGB]{0, 0, 0}\vrule width 0pt}}{\hspace*{4pt}\rule{0pt}{\baselineskip+4pt}\raggedleft {\fontsize{10pt}{12pt}\selectfont -29.650 *~~}\rule[-4pt]{0pt}{4pt}\hspace*{4pt}} & 
\multicolumn{1}{r!{\color[RGB]{0, 0, 0}\vrule width 0pt}}{\hspace*{4pt}\rule{0pt}{\baselineskip+4pt}\raggedleft {\fontsize{10pt}{12pt}\selectfont -102.165 ***}\rule[-4pt]{0pt}{4pt}\hspace*{4pt}} & 
\multicolumn{1}{r!{\color[RGB]{0, 0, 0}\vrule width 0pt}}{\hspace*{4pt}\rule{0pt}{\baselineskip+4pt}\raggedleft {\fontsize{10pt}{12pt}\selectfont 90.043 ***}\rule[-4pt]{0pt}{4pt}\hspace*{4pt}}\tabularnewline[-0.5pt]
\multicolumn{1}{!{\color[RGB]{0, 0, 0}\vrule width 0pt}l!{\color[RGB]{0, 0, 0}\vrule width 0pt}}{\hspace*{4pt}\rule{0pt}{\baselineskip+4pt}\raggedright {\fontsize{10pt}{12pt}\selectfont }\rule[-4pt]{0pt}{4pt}\hspace*{4pt}} & 
\multicolumn{1}{r!{\color[RGB]{0, 0, 0}\vrule width 0pt}}{\hspace*{4pt}\rule{0pt}{\baselineskip+4pt}\raggedleft {\fontsize{10pt}{12pt}\selectfont ~~~~~~~~}\rule[-4pt]{0pt}{4pt}\hspace*{4pt}} & 
\multicolumn{1}{r!{\color[RGB]{0, 0, 0}\vrule width 0pt}}{\hspace*{4pt}\rule{0pt}{\baselineskip+4pt}\raggedleft {\fontsize{10pt}{12pt}\selectfont (11.990)~~~}\rule[-4pt]{0pt}{4pt}\hspace*{4pt}} & 
\multicolumn{1}{r!{\color[RGB]{0, 0, 0}\vrule width 0pt}}{\hspace*{4pt}\rule{0pt}{\baselineskip+4pt}\raggedleft {\fontsize{10pt}{12pt}\selectfont (4.635)~~~}\rule[-4pt]{0pt}{4pt}\hspace*{4pt}} & 
\multicolumn{1}{r!{\color[RGB]{0, 0, 0}\vrule width 0pt}}{\hspace*{4pt}\rule{0pt}{\baselineskip+4pt}\raggedleft {\fontsize{10pt}{12pt}\selectfont (9.588)~~~}\rule[-4pt]{0pt}{4pt}\hspace*{4pt}}\tabularnewline[-0.5pt]
\multicolumn{1}{!{\color[RGB]{0, 0, 0}\vrule width 0pt}l!{\color[RGB]{0, 0, 0}\vrule width 0pt}}{\hspace*{4pt}\rule{0pt}{\baselineskip+4pt}\raggedright {\fontsize{10pt}{12pt}\selectfont carat:depth}\rule[-4pt]{0pt}{4pt}\hspace*{4pt}} & 
\multicolumn{1}{r!{\color[RGB]{0, 0, 0}\vrule width 0pt}}{\hspace*{4pt}\rule{0pt}{\baselineskip+4pt}\raggedleft {\fontsize{10pt}{12pt}\selectfont ~~~~~~~~}\rule[-4pt]{0pt}{4pt}\hspace*{4pt}} & 
\multicolumn{1}{r!{\color[RGB]{0, 0, 0}\vrule width 0pt}}{\hspace*{4pt}\rule{0pt}{\baselineskip+4pt}\raggedleft {\fontsize{10pt}{12pt}\selectfont ~~~~~~~~}\rule[-4pt]{0pt}{4pt}\hspace*{4pt}} & 
\multicolumn{1}{r!{\color[RGB]{0, 0, 0}\vrule width 0pt}}{\hspace*{4pt}\rule{0pt}{\baselineskip+4pt}\raggedleft {\fontsize{10pt}{12pt}\selectfont ~~~~~~~~}\rule[-4pt]{0pt}{4pt}\hspace*{4pt}} & 
\multicolumn{1}{r!{\color[RGB]{0, 0, 0}\vrule width 0pt}}{\hspace*{4pt}\rule{0pt}{\baselineskip+4pt}\raggedleft {\fontsize{10pt}{12pt}\selectfont -210.075 ***}\rule[-4pt]{0pt}{4pt}\hspace*{4pt}}\tabularnewline[-0.5pt]
\multicolumn{1}{!{\color[RGB]{0, 0, 0}\vrule width 0pt}l!{\color[RGB]{0, 0, 0}\vrule width 0pt}}{\hspace*{4pt}\rule{0pt}{\baselineskip+4pt}\raggedright {\fontsize{10pt}{12pt}\selectfont }\rule[-4pt]{0pt}{4pt}\hspace*{4pt}} & 
\multicolumn{1}{r!{\color[RGB]{0, 0, 0}\vrule width 0pt}}{\hspace*{4pt}\rule{0pt}{\baselineskip+4pt}\raggedleft {\fontsize{10pt}{12pt}\selectfont ~~~~~~~~}\rule[-4pt]{0pt}{4pt}\hspace*{4pt}} & 
\multicolumn{1}{r!{\color[RGB]{0, 0, 0}\vrule width 0pt}}{\hspace*{4pt}\rule{0pt}{\baselineskip+4pt}\raggedleft {\fontsize{10pt}{12pt}\selectfont ~~~~~~~~}\rule[-4pt]{0pt}{4pt}\hspace*{4pt}} & 
\multicolumn{1}{r!{\color[RGB]{0, 0, 0}\vrule width 0pt}}{\hspace*{4pt}\rule{0pt}{\baselineskip+4pt}\raggedleft {\fontsize{10pt}{12pt}\selectfont ~~~~~~~~}\rule[-4pt]{0pt}{4pt}\hspace*{4pt}} & 
\multicolumn{1}{r!{\color[RGB]{0, 0, 0}\vrule width 0pt}}{\hspace*{4pt}\rule{0pt}{\baselineskip+4pt}\raggedleft {\fontsize{10pt}{12pt}\selectfont (9.187)~~~}\rule[-4pt]{0pt}{4pt}\hspace*{4pt}}\tabularnewline[-0.5pt]


\hhline{>{\arrayrulecolor[RGB]{0, 0, 0}\global\arrayrulewidth=0.4pt}->{\arrayrulecolor[RGB]{0, 0, 0}\global\arrayrulewidth=0.4pt}->{\arrayrulecolor[RGB]{0, 0, 0}\global\arrayrulewidth=0.4pt}->{\arrayrulecolor[RGB]{0, 0, 0}\global\arrayrulewidth=0.4pt}->{\arrayrulecolor[RGB]{0, 0, 0}\global\arrayrulewidth=0.4pt}-}
\arrayrulecolor{black}
\multicolumn{1}{!{\color[RGB]{0, 0, 0}\vrule width 0pt}l!{\color[RGB]{0, 0, 0}\vrule width 0pt}}{\hspace*{4pt}\rule{0pt}{\baselineskip+4pt}\raggedright {\fontsize{10pt}{12pt}\selectfont N}\rule[-4pt]{0pt}{4pt}\hspace*{4pt}} & 
\multicolumn{1}{r!{\color[RGB]{0, 0, 0}\vrule width 0pt}}{\hspace*{4pt}\rule{0pt}{\baselineskip+4pt}\raggedleft {\fontsize{10pt}{12pt}\selectfont 53940~~~~~~~~}\rule[-4pt]{0pt}{4pt}\hspace*{4pt}} & 
\multicolumn{1}{r!{\color[RGB]{0, 0, 0}\vrule width 0pt}}{\hspace*{4pt}\rule{0pt}{\baselineskip+4pt}\raggedleft {\fontsize{10pt}{12pt}\selectfont 53940~~~~~~~~}\rule[-4pt]{0pt}{4pt}\hspace*{4pt}} & 
\multicolumn{1}{r!{\color[RGB]{0, 0, 0}\vrule width 0pt}}{\hspace*{4pt}\rule{0pt}{\baselineskip+4pt}\raggedleft {\fontsize{10pt}{12pt}\selectfont 53940~~~~~~~~}\rule[-4pt]{0pt}{4pt}\hspace*{4pt}} & 
\multicolumn{1}{r!{\color[RGB]{0, 0, 0}\vrule width 0pt}}{\hspace*{4pt}\rule{0pt}{\baselineskip+4pt}\raggedleft {\fontsize{10pt}{12pt}\selectfont 53940~~~~~~~~}\rule[-4pt]{0pt}{4pt}\hspace*{4pt}}\tabularnewline[-0.5pt]
\multicolumn{1}{!{\color[RGB]{0, 0, 0}\vrule width 0pt}l!{\color[RGB]{0, 0, 0}\vrule width 0pt}}{\hspace*{4pt}\rule{0pt}{\baselineskip+4pt}\raggedright {\fontsize{10pt}{12pt}\selectfont R2}\rule[-4pt]{0pt}{4pt}\hspace*{4pt}} & 
\multicolumn{1}{r!{\color[RGB]{0, 0, 0}\vrule width 0pt}}{\hspace*{4pt}\rule{0pt}{\baselineskip+4pt}\raggedleft {\fontsize{10pt}{12pt}\selectfont 0.849~~~~}\rule[-4pt]{0pt}{4pt}\hspace*{4pt}} & 
\multicolumn{1}{r!{\color[RGB]{0, 0, 0}\vrule width 0pt}}{\hspace*{4pt}\rule{0pt}{\baselineskip+4pt}\raggedleft {\fontsize{10pt}{12pt}\selectfont 0.000~~~~}\rule[-4pt]{0pt}{4pt}\hspace*{4pt}} & 
\multicolumn{1}{r!{\color[RGB]{0, 0, 0}\vrule width 0pt}}{\hspace*{4pt}\rule{0pt}{\baselineskip+4pt}\raggedleft {\fontsize{10pt}{12pt}\selectfont 0.851~~~~}\rule[-4pt]{0pt}{4pt}\hspace*{4pt}} & 
\multicolumn{1}{r!{\color[RGB]{0, 0, 0}\vrule width 0pt}}{\hspace*{4pt}\rule{0pt}{\baselineskip+4pt}\raggedleft {\fontsize{10pt}{12pt}\selectfont 0.852~~~~}\rule[-4pt]{0pt}{4pt}\hspace*{4pt}}\tabularnewline[-0.5pt]


\hhline{>{\arrayrulecolor[RGB]{0, 0, 0}\global\arrayrulewidth=0.4pt}->{\arrayrulecolor[RGB]{0, 0, 0}\global\arrayrulewidth=0.4pt}->{\arrayrulecolor[RGB]{0, 0, 0}\global\arrayrulewidth=0.4pt}->{\arrayrulecolor[RGB]{0, 0, 0}\global\arrayrulewidth=0.4pt}->{\arrayrulecolor[RGB]{0, 0, 0}\global\arrayrulewidth=0.4pt}-}
\arrayrulecolor{black}
\multicolumn{5}{!{\color[RGB]{0, 0, 0}\vrule width 0pt}p{0.6\textwidth+8\tabcolsep}!{\color[RGB]{0, 0, 0}\vrule width 0pt}}{\hspace*{4pt}\parbox[b]{0.6\textwidth+8\tabcolsep-4pt-4pt}{\rule{0pt}{\baselineskip+4pt}\raggedright {\fontsize{10pt}{12pt}\selectfont *** p $<$ 0.001; ** p $<$ 0.01; * p $<$ 0.05.}\rule[-4pt]{0pt}{4pt}}\hspace*{4pt}}\tabularnewline[-0.5pt]
\end{tabularx}

\end{table}

From Table \ref{Reg01} above, we see how to make reporting regressions
easy. Making nice tables, though, takes some practice and there are
various other packages you can explore as well.

\section{Lists}\label{lists}

To add lists, simply using the following notation

\begin{itemize}
\item
  This is really simple

  \begin{itemize}
  \tightlist
  \item
    Just note the spaces here - writing in R you have to sometimes be
    pedantic about spaces\ldots{}
  \end{itemize}
\item
  Note that Rmarkdown notation removes the pain of defining
  \LaTeX environments!
\end{itemize}

\section{Conclusion}\label{conclusion}

I hope you find this template useful. Remember, stackoverflow is your
friend - use it to find answers to questions. Feel free to write me a
mail if you have any questions regarding the use of this package. To
cite this package, simply type citation(``Texevier'') in Rstudio to get
the citation for Katzke (\protect\hyperlink{ref-Texevier}{2017}) (Note
that uncited references in your bibtex file will not be included in
References).

\section*{References}\label{references}
\addcontentsline{toc}{section}{References}

\hypertarget{refs}{}
\hypertarget{ref-fama1997}{}
Fama, Eugene F, and Kenneth R French. 1997. ``Industry Costs of
Equity.'' \emph{Journal of Financial Economics} 43 (2). Elsevier:
153--93.

\hypertarget{ref-grinold2000}{}
Grinold, Richard C, and Ronald N Kahn. 2000. ``Active Portfolio
Management.'' McGraw Hill New York, NY.

\hypertarget{ref-Texevier}{}
Katzke, N.F. 2017. \emph{Texevier: Package to Create Elsevier Templates
for Rmarkdown}. Stellenbosch, South Africa: Bureau for Economic
Research.

% Force include bibliography in my chosen format:
\newpage
\nocite{*}
\bibliography{}





\end{document}
